\documentclass[]{article}
\usepackage{lmodern}
\usepackage{amssymb,amsmath}
\usepackage{ifxetex,ifluatex}
\usepackage{fixltx2e} % provides \textsubscript
\ifnum 0\ifxetex 1\fi\ifluatex 1\fi=0 % if pdftex
  \usepackage[T1]{fontenc}
  \usepackage[utf8]{inputenc}
\else % if luatex or xelatex
  \ifxetex
    \usepackage{mathspec}
  \else
    \usepackage{fontspec}
  \fi
  \defaultfontfeatures{Ligatures=TeX,Scale=MatchLowercase}
\fi
% use upquote if available, for straight quotes in verbatim environments
\IfFileExists{upquote.sty}{\usepackage{upquote}}{}
% use microtype if available
\IfFileExists{microtype.sty}{%
\usepackage{microtype}
\UseMicrotypeSet[protrusion]{basicmath} % disable protrusion for tt fonts
}{}
\usepackage[margin=1in]{geometry}
\usepackage{hyperref}
\hypersetup{unicode=true,
            pdftitle={DSII\_homework2},
            pdfauthor={Chirag Shah},
            pdfborder={0 0 0},
            breaklinks=true}
\urlstyle{same}  % don't use monospace font for urls
\usepackage{color}
\usepackage{fancyvrb}
\newcommand{\VerbBar}{|}
\newcommand{\VERB}{\Verb[commandchars=\\\{\}]}
\DefineVerbatimEnvironment{Highlighting}{Verbatim}{commandchars=\\\{\}}
% Add ',fontsize=\small' for more characters per line
\usepackage{framed}
\definecolor{shadecolor}{RGB}{248,248,248}
\newenvironment{Shaded}{\begin{snugshade}}{\end{snugshade}}
\newcommand{\KeywordTok}[1]{\textcolor[rgb]{0.13,0.29,0.53}{\textbf{#1}}}
\newcommand{\DataTypeTok}[1]{\textcolor[rgb]{0.13,0.29,0.53}{#1}}
\newcommand{\DecValTok}[1]{\textcolor[rgb]{0.00,0.00,0.81}{#1}}
\newcommand{\BaseNTok}[1]{\textcolor[rgb]{0.00,0.00,0.81}{#1}}
\newcommand{\FloatTok}[1]{\textcolor[rgb]{0.00,0.00,0.81}{#1}}
\newcommand{\ConstantTok}[1]{\textcolor[rgb]{0.00,0.00,0.00}{#1}}
\newcommand{\CharTok}[1]{\textcolor[rgb]{0.31,0.60,0.02}{#1}}
\newcommand{\SpecialCharTok}[1]{\textcolor[rgb]{0.00,0.00,0.00}{#1}}
\newcommand{\StringTok}[1]{\textcolor[rgb]{0.31,0.60,0.02}{#1}}
\newcommand{\VerbatimStringTok}[1]{\textcolor[rgb]{0.31,0.60,0.02}{#1}}
\newcommand{\SpecialStringTok}[1]{\textcolor[rgb]{0.31,0.60,0.02}{#1}}
\newcommand{\ImportTok}[1]{#1}
\newcommand{\CommentTok}[1]{\textcolor[rgb]{0.56,0.35,0.01}{\textit{#1}}}
\newcommand{\DocumentationTok}[1]{\textcolor[rgb]{0.56,0.35,0.01}{\textbf{\textit{#1}}}}
\newcommand{\AnnotationTok}[1]{\textcolor[rgb]{0.56,0.35,0.01}{\textbf{\textit{#1}}}}
\newcommand{\CommentVarTok}[1]{\textcolor[rgb]{0.56,0.35,0.01}{\textbf{\textit{#1}}}}
\newcommand{\OtherTok}[1]{\textcolor[rgb]{0.56,0.35,0.01}{#1}}
\newcommand{\FunctionTok}[1]{\textcolor[rgb]{0.00,0.00,0.00}{#1}}
\newcommand{\VariableTok}[1]{\textcolor[rgb]{0.00,0.00,0.00}{#1}}
\newcommand{\ControlFlowTok}[1]{\textcolor[rgb]{0.13,0.29,0.53}{\textbf{#1}}}
\newcommand{\OperatorTok}[1]{\textcolor[rgb]{0.81,0.36,0.00}{\textbf{#1}}}
\newcommand{\BuiltInTok}[1]{#1}
\newcommand{\ExtensionTok}[1]{#1}
\newcommand{\PreprocessorTok}[1]{\textcolor[rgb]{0.56,0.35,0.01}{\textit{#1}}}
\newcommand{\AttributeTok}[1]{\textcolor[rgb]{0.77,0.63,0.00}{#1}}
\newcommand{\RegionMarkerTok}[1]{#1}
\newcommand{\InformationTok}[1]{\textcolor[rgb]{0.56,0.35,0.01}{\textbf{\textit{#1}}}}
\newcommand{\WarningTok}[1]{\textcolor[rgb]{0.56,0.35,0.01}{\textbf{\textit{#1}}}}
\newcommand{\AlertTok}[1]{\textcolor[rgb]{0.94,0.16,0.16}{#1}}
\newcommand{\ErrorTok}[1]{\textcolor[rgb]{0.64,0.00,0.00}{\textbf{#1}}}
\newcommand{\NormalTok}[1]{#1}
\usepackage{graphicx,grffile}
\makeatletter
\def\maxwidth{\ifdim\Gin@nat@width>\linewidth\linewidth\else\Gin@nat@width\fi}
\def\maxheight{\ifdim\Gin@nat@height>\textheight\textheight\else\Gin@nat@height\fi}
\makeatother
% Scale images if necessary, so that they will not overflow the page
% margins by default, and it is still possible to overwrite the defaults
% using explicit options in \includegraphics[width, height, ...]{}
\setkeys{Gin}{width=\maxwidth,height=\maxheight,keepaspectratio}
\IfFileExists{parskip.sty}{%
\usepackage{parskip}
}{% else
\setlength{\parindent}{0pt}
\setlength{\parskip}{6pt plus 2pt minus 1pt}
}
\setlength{\emergencystretch}{3em}  % prevent overfull lines
\providecommand{\tightlist}{%
  \setlength{\itemsep}{0pt}\setlength{\parskip}{0pt}}
\setcounter{secnumdepth}{0}
% Redefines (sub)paragraphs to behave more like sections
\ifx\paragraph\undefined\else
\let\oldparagraph\paragraph
\renewcommand{\paragraph}[1]{\oldparagraph{#1}\mbox{}}
\fi
\ifx\subparagraph\undefined\else
\let\oldsubparagraph\subparagraph
\renewcommand{\subparagraph}[1]{\oldsubparagraph{#1}\mbox{}}
\fi

%%% Use protect on footnotes to avoid problems with footnotes in titles
\let\rmarkdownfootnote\footnote%
\def\footnote{\protect\rmarkdownfootnote}

%%% Change title format to be more compact
\usepackage{titling}

% Create subtitle command for use in maketitle
\newcommand{\subtitle}[1]{
  \posttitle{
    \begin{center}\large#1\end{center}
    }
}

\setlength{\droptitle}{-2em}

  \title{DSII\_homework2}
    \pretitle{\vspace{\droptitle}\centering\huge}
  \posttitle{\par}
    \author{Chirag Shah}
    \preauthor{\centering\large\emph}
  \postauthor{\par}
      \predate{\centering\large\emph}
  \postdate{\par}
    \date{2019-03-19}


\begin{document}
\maketitle

Data Import

\begin{Shaded}
\begin{Highlighting}[]
\NormalTok{concrete =}\StringTok{ }\KeywordTok{read_csv}\NormalTok{(}\StringTok{"concrete.csv"}\NormalTok{) }\OperatorTok
\StringTok{  }\NormalTok{janitor}\OperatorTok{::}\KeywordTok{clean_names}\NormalTok{()}
\end{Highlighting}
\end{Shaded}

\begin{verbatim}
## Parsed with column specification:
## cols(
##   Cement = col_double(),
##   BlastFurnaceSlag = col_double(),
##   FlyAsh = col_double(),
##   Water = col_double(),
##   Superplasticizer = col_double(),
##   CoarseAggregate = col_double(),
##   FineAggregate = col_double(),
##   Age = col_integer(),
##   CompressiveStrength = col_double()
## )
\end{verbatim}

\section{Part A}\label{part-a}

\begin{Shaded}
\begin{Highlighting}[]
\CommentTok{# matrix of predictors }
\NormalTok{x <-}\StringTok{ }\KeywordTok{model.matrix}\NormalTok{(compressive_strength}\OperatorTok{~}\NormalTok{.,concrete)[,}\OperatorTok{-}\DecValTok{1}\NormalTok{]}
\CommentTok{# vector of response}
\NormalTok{y <-}\StringTok{ }\NormalTok{concrete}\OperatorTok{$}\NormalTok{compressive_strength}
\NormalTok{theme1 <-}\StringTok{ }\KeywordTok{trellis.par.get}\NormalTok{()}
\NormalTok{theme1}\OperatorTok{$}\NormalTok{plot.symbol}\OperatorTok{$}\NormalTok{col <-}\StringTok{ }\KeywordTok{rgb}\NormalTok{(.}\DecValTok{2}\NormalTok{, .}\DecValTok{4}\NormalTok{, .}\DecValTok{2}\NormalTok{, .}\DecValTok{5}\NormalTok{)}
\NormalTok{theme1}\OperatorTok{$}\NormalTok{plot.symbol}\OperatorTok{$}\NormalTok{pch <-}\StringTok{ }\DecValTok{16}
\NormalTok{theme1}\OperatorTok{$}\NormalTok{plot.line}\OperatorTok{$}\NormalTok{col <-}\StringTok{ }\KeywordTok{rgb}\NormalTok{(.}\DecValTok{8}\NormalTok{, .}\DecValTok{1}\NormalTok{, .}\DecValTok{1}\NormalTok{, }\DecValTok{1}\NormalTok{)}
\NormalTok{theme1}\OperatorTok{$}\NormalTok{plot.line}\OperatorTok{$}\NormalTok{lwd <-}\StringTok{ }\DecValTok{2}
\NormalTok{theme1}\OperatorTok{$}\NormalTok{strip.background}\OperatorTok{$}\NormalTok{col <-}\StringTok{ }\KeywordTok{rgb}\NormalTok{(.}\DecValTok{0}\NormalTok{, .}\DecValTok{2}\NormalTok{, .}\DecValTok{6}\NormalTok{, .}\DecValTok{2}\NormalTok{)}
\KeywordTok{trellis.par.set}\NormalTok{(theme1)}
\KeywordTok{featurePlot}\NormalTok{(x, y, }\DataTypeTok{plot =} \StringTok{"scatter"}\NormalTok{, }\DataTypeTok{labels =} \KeywordTok{c}\NormalTok{(}\StringTok{""}\NormalTok{,}\StringTok{"compressive strength"}\NormalTok{),}
            \DataTypeTok{type =} \KeywordTok{c}\NormalTok{(}\StringTok{"p"}\NormalTok{), }\DataTypeTok{layout =} \KeywordTok{c}\NormalTok{(}\DecValTok{4}\NormalTok{, }\DecValTok{2}\NormalTok{))}
\end{Highlighting}
\end{Shaded}

\includegraphics{dsII_hw2_cs3652_files/figure-latex/unnamed-chunk-2-1.pdf}

\section{Part B}\label{part-b}

\begin{Shaded}
\begin{Highlighting}[]
\CommentTok{#polynomial regression}
\NormalTok{fit1 <-}\StringTok{ }\KeywordTok{lm}\NormalTok{(compressive_strength}\OperatorTok{~}\NormalTok{water, }\DataTypeTok{data =}\NormalTok{ concrete)}
\NormalTok{fit2 <-}\StringTok{ }\KeywordTok{lm}\NormalTok{(compressive_strength}\OperatorTok{~}\KeywordTok{poly}\NormalTok{(water,}\DecValTok{2}\NormalTok{), }\DataTypeTok{data =}\NormalTok{ concrete) }
\NormalTok{fit3 <-}\StringTok{ }\KeywordTok{lm}\NormalTok{(compressive_strength}\OperatorTok{~}\KeywordTok{poly}\NormalTok{(water,}\DecValTok{3}\NormalTok{), }\DataTypeTok{data =}\NormalTok{ concrete) }
\NormalTok{fit4 <-}\StringTok{ }\KeywordTok{lm}\NormalTok{(compressive_strength}\OperatorTok{~}\KeywordTok{poly}\NormalTok{(water,}\DecValTok{4}\NormalTok{), }\DataTypeTok{data =}\NormalTok{ concrete) }
\end{Highlighting}
\end{Shaded}

\begin{Shaded}
\begin{Highlighting}[]
\KeywordTok{set.seed}\NormalTok{(}\DecValTok{1}\NormalTok{)}

\NormalTok{mseK <-}\StringTok{ }\KeywordTok{rep}\NormalTok{(}\OtherTok{NA}\NormalTok{, }\DecValTok{4}\NormalTok{)}
\ControlFlowTok{for}\NormalTok{ (i }\ControlFlowTok{in} \DecValTok{1}\OperatorTok{:}\DecValTok{4}\NormalTok{) \{}
\NormalTok{    fit <-}\StringTok{ }\KeywordTok{glm}\NormalTok{(compressive_strength }\OperatorTok{~}\StringTok{ }\KeywordTok{poly}\NormalTok{(water, i), }\DataTypeTok{data =}\NormalTok{ concrete)}
\NormalTok{    mseK[i] <-}\StringTok{ }\KeywordTok{cv.glm}\NormalTok{(concrete, fit, }\DataTypeTok{K =} \DecValTok{10}\NormalTok{)}\OperatorTok{$}\NormalTok{delta[}\DecValTok{1}\NormalTok{]}
\NormalTok{\}}

\KeywordTok{plot}\NormalTok{(}\DecValTok{1}\OperatorTok{:}\DecValTok{4}\NormalTok{, mseK, }\DataTypeTok{xlab =} \StringTok{"Exponent"}\NormalTok{, }\DataTypeTok{ylab =} \StringTok{"Test MSE"}\NormalTok{, }\DataTypeTok{type =} \StringTok{"l"}\NormalTok{)}
\end{Highlighting}
\end{Shaded}

\includegraphics{dsII_hw2_cs3652_files/figure-latex/unnamed-chunk-4-1.pdf}

The cross validation approach produced models with very similar RMSE
values. While models with the exponent of 2 and 3 for water are the
lowest they aren't by much. With that being said, the r squared value
for the exponent 4 is the highest indicating that the differences in
compressive\_strength is more explained by water\^{}4 than any other
polynomial function. Using the for loop, the cross validation provided
an exponent of 4 as the best polynomial function. Therefore lets confirm
using the anova test.

\begin{Shaded}
\begin{Highlighting}[]
\KeywordTok{anova}\NormalTok{(fit1,fit2,fit3,fit4) }
\end{Highlighting}
\end{Shaded}

\begin{verbatim}
## Analysis of Variance Table
## 
## Model 1: compressive_strength ~ water
## Model 2: compressive_strength ~ poly(water, 2)
## Model 3: compressive_strength ~ poly(water, 3)
## Model 4: compressive_strength ~ poly(water, 4)
##   Res.Df    RSS Df Sum of Sq      F    Pr(>F)    
## 1   1028 263085                                  
## 2   1027 247712  1   15372.8 68.140 4.652e-16 ***
## 3   1026 235538  1   12174.0 53.962 4.166e-13 ***
## 4   1025 231246  1    4291.5 19.022 1.423e-05 ***
## ---
## Signif. codes:  0 '***' 0.001 '**' 0.01 '*' 0.05 '.' 0.1 ' ' 1
\end{verbatim}

\begin{Shaded}
\begin{Highlighting}[]
\NormalTok{plot1 <-}\StringTok{ }\KeywordTok{ggplot}\NormalTok{(}\DataTypeTok{data =}\NormalTok{ concrete, }\KeywordTok{aes}\NormalTok{(}\DataTypeTok{x =}\NormalTok{ water, }\DataTypeTok{y =}\NormalTok{ compressive_strength)) }\OperatorTok{+}
\StringTok{     }\KeywordTok{geom_point}\NormalTok{(}\DataTypeTok{color =} \KeywordTok{rgb}\NormalTok{(.}\DecValTok{2}\NormalTok{, .}\DecValTok{4}\NormalTok{, .}\DecValTok{2}\NormalTok{, .}\DecValTok{5}\NormalTok{))}
\NormalTok{plot1}
\end{Highlighting}
\end{Shaded}

\includegraphics{dsII_hw2_cs3652_files/figure-latex/unnamed-chunk-5-1.pdf}

Using the anova test, models with an exponent of 2, 3, and 4 are
significant. However, lets use d=2 since we want the most parsimonious
model.

\begin{Shaded}
\begin{Highlighting}[]
\KeywordTok{plot}\NormalTok{(compressive_strength }\OperatorTok{~}\StringTok{ }\NormalTok{water, }\DataTypeTok{data =}\NormalTok{ concrete, }\DataTypeTok{col =} \StringTok{"blue"}\NormalTok{)}
\NormalTok{waterlims <-}\StringTok{ }\KeywordTok{range}\NormalTok{(concrete}\OperatorTok{$}\NormalTok{water)}
\NormalTok{water.grid <-}\StringTok{ }\KeywordTok{seq}\NormalTok{(}\DataTypeTok{from =}\NormalTok{ waterlims[}\DecValTok{1}\NormalTok{], }\DataTypeTok{to =}\NormalTok{ waterlims[}\DecValTok{2}\NormalTok{], }\DataTypeTok{by =} \DecValTok{1}\NormalTok{)}

\NormalTok{preds1 <-}\StringTok{ }\KeywordTok{predict}\NormalTok{(fit1, }\DataTypeTok{newdata =} \KeywordTok{data.frame}\NormalTok{(}\DataTypeTok{water =}\NormalTok{ water.grid))}
\NormalTok{preds2 <-}\StringTok{ }\KeywordTok{predict}\NormalTok{(fit2, }\DataTypeTok{newdata =} \KeywordTok{data.frame}\NormalTok{(}\DataTypeTok{water =}\NormalTok{ water.grid))}
\NormalTok{preds3 <-}\StringTok{ }\KeywordTok{predict}\NormalTok{(fit3, }\DataTypeTok{newdata =} \KeywordTok{data.frame}\NormalTok{(}\DataTypeTok{water =}\NormalTok{ water.grid))}
\NormalTok{preds4 <-}\StringTok{ }\KeywordTok{predict}\NormalTok{(fit4, }\DataTypeTok{newdata =} \KeywordTok{data.frame}\NormalTok{(}\DataTypeTok{water =}\NormalTok{ water.grid))}

\KeywordTok{lines}\NormalTok{(water.grid, preds1, }\DataTypeTok{col =} \StringTok{"red"}\NormalTok{, }\DataTypeTok{lwd =} \DecValTok{2}\NormalTok{)}
\KeywordTok{lines}\NormalTok{(water.grid, preds2, }\DataTypeTok{col =} \StringTok{"purple"}\NormalTok{, }\DataTypeTok{lwd =} \DecValTok{2}\NormalTok{)}
\KeywordTok{lines}\NormalTok{(water.grid, preds3, }\DataTypeTok{col =} \StringTok{"green"}\NormalTok{, }\DataTypeTok{lwd =} \DecValTok{2}\NormalTok{)}
\KeywordTok{lines}\NormalTok{(water.grid, preds4, }\DataTypeTok{col =} \StringTok{"brown"}\NormalTok{, }\DataTypeTok{lwd =} \DecValTok{2}\NormalTok{)}
\end{Highlighting}
\end{Shaded}

\includegraphics{dsII_hw2_cs3652_files/figure-latex/unnamed-chunk-6-1.pdf}

\section{Part C}\label{part-c}

\begin{Shaded}
\begin{Highlighting}[]
\KeywordTok{plot}\NormalTok{(compressive_strength }\OperatorTok{~}\StringTok{ }\NormalTok{water, }\DataTypeTok{data =}\NormalTok{ concrete, }\DataTypeTok{col =} \StringTok{"red"}\NormalTok{)}

\NormalTok{waterlims <-}\StringTok{ }\KeywordTok{range}\NormalTok{(concrete}\OperatorTok{$}\NormalTok{water)}
\NormalTok{water.grid <-}\StringTok{ }\KeywordTok{seq}\NormalTok{(}\DataTypeTok{from =}\NormalTok{ waterlims[}\DecValTok{1}\NormalTok{], }\DataTypeTok{to =}\NormalTok{ waterlims[}\DecValTok{2}\NormalTok{], }\DataTypeTok{by =} \DecValTok{1}\NormalTok{)}
\NormalTok{fit <-}\StringTok{ }\KeywordTok{lm}\NormalTok{(compressive_strength }\OperatorTok{~}\StringTok{ }\KeywordTok{poly}\NormalTok{(water, }\DecValTok{4}\NormalTok{), }\DataTypeTok{data =}\NormalTok{ concrete)}
\NormalTok{preds <-}\StringTok{ }\KeywordTok{predict}\NormalTok{(fit, }\DataTypeTok{newdata =} \KeywordTok{data.frame}\NormalTok{(}\DataTypeTok{water =}\NormalTok{ water.grid))}
\KeywordTok{lines}\NormalTok{(water.grid, preds, }\DataTypeTok{col =} \StringTok{"blue"}\NormalTok{, }\DataTypeTok{lwd =} \DecValTok{2}\NormalTok{)}
\end{Highlighting}
\end{Shaded}

\includegraphics{dsII_hw2_cs3652_files/figure-latex/unnamed-chunk-7-1.pdf}

\begin{Shaded}
\begin{Highlighting}[]
\NormalTok{fit.ss <-}\StringTok{ }\KeywordTok{smooth.spline}\NormalTok{(concrete}\OperatorTok{$}\NormalTok{water, concrete}\OperatorTok{$}\NormalTok{compressive_strength)}
\NormalTok{fit.ss}\OperatorTok{$}\NormalTok{df}
\end{Highlighting}
\end{Shaded}

\begin{verbatim}
## [1] 68.88205
\end{verbatim}

\begin{Shaded}
\begin{Highlighting}[]
\NormalTok{pred.ss <-}\StringTok{ }\KeywordTok{predict}\NormalTok{(fit.ss, }\DataTypeTok{x =}\NormalTok{ water.grid)}
\NormalTok{pred.ss.df <-}\StringTok{ }\KeywordTok{data.frame}\NormalTok{(}\DataTypeTok{pred =}\NormalTok{ pred.ss}\OperatorTok{$}\NormalTok{y,}
                         \DataTypeTok{water =}\NormalTok{ water.grid)}

\NormalTok{p <-}\StringTok{ }\KeywordTok{ggplot}\NormalTok{(}\DataTypeTok{data =}\NormalTok{ concrete, }\KeywordTok{aes}\NormalTok{(}\DataTypeTok{x =}\NormalTok{ water, }\DataTypeTok{y =}\NormalTok{ compressive_strength)) }\OperatorTok{+}
\StringTok{  }\KeywordTok{geom_point}\NormalTok{(}\DataTypeTok{color =} \KeywordTok{rgb}\NormalTok{(.}\DecValTok{2}\NormalTok{, .}\DecValTok{4}\NormalTok{, .}\DecValTok{2}\NormalTok{, .}\DecValTok{5}\NormalTok{))}
\NormalTok{p }\OperatorTok{+}\StringTok{ }\KeywordTok{geom_line}\NormalTok{(}\KeywordTok{aes}\NormalTok{(}\DataTypeTok{x =}\NormalTok{ water, }\DataTypeTok{y =}\NormalTok{ pred), }\DataTypeTok{data =}\NormalTok{ pred.ss.df, }
              \DataTypeTok{color =} \KeywordTok{rgb}\NormalTok{(.}\DecValTok{8}\NormalTok{, .}\DecValTok{1}\NormalTok{, .}\DecValTok{1}\NormalTok{, }\DecValTok{1}\NormalTok{)) }\OperatorTok{+}\StringTok{ }\KeywordTok{theme_bw}\NormalTok{() }\OperatorTok{+}\StringTok{ }
\StringTok{  }\KeywordTok{labs}\NormalTok{(}\DataTypeTok{title =} \StringTok{'Select degrees of freedom using GCV'}\NormalTok{) }\OperatorTok{+}\StringTok{ }
\StringTok{  }\KeywordTok{theme}\NormalTok{(}\DataTypeTok{plot.title =} \KeywordTok{element_text}\NormalTok{(}\DataTypeTok{hjust =} \FloatTok{0.5}\NormalTok{))}
\end{Highlighting}
\end{Shaded}

\includegraphics{dsII_hw2_cs3652_files/figure-latex/unnamed-chunk-8-1.pdf}

Here we will fit for different degrees of freedom for 2 to 10.

\begin{Shaded}
\begin{Highlighting}[]
\KeywordTok{par}\NormalTok{(}\DataTypeTok{mfrow =} \KeywordTok{c}\NormalTok{(}\DecValTok{3}\NormalTok{,}\DecValTok{3}\NormalTok{)) }\CommentTok{# 3 x 3 grid}
\NormalTok{all.dfs =}\StringTok{ }\KeywordTok{rep}\NormalTok{(}\OtherTok{NA}\NormalTok{, }\DecValTok{9}\NormalTok{)}
\ControlFlowTok{for}\NormalTok{ (i }\ControlFlowTok{in} \DecValTok{2}\OperatorTok{:}\DecValTok{10}\NormalTok{) \{}
\NormalTok{  fit.ss =}\StringTok{ }\KeywordTok{smooth.spline}\NormalTok{(concrete}\OperatorTok{$}\NormalTok{water, concrete}\OperatorTok{$}\NormalTok{compressive_strength, }\DataTypeTok{df =}\NormalTok{ i)}
  
\NormalTok{  pred.ss <-}\StringTok{ }\KeywordTok{predict}\NormalTok{(fit.ss, }\DataTypeTok{x =}\NormalTok{ water.grid)}
  
  \KeywordTok{plot}\NormalTok{(concrete}\OperatorTok{$}\NormalTok{water, concrete}\OperatorTok{$}\NormalTok{compressive_strength, }\DataTypeTok{cex =}\NormalTok{ .}\DecValTok{5}\NormalTok{, }\DataTypeTok{col =} \StringTok{"darkgrey"}\NormalTok{)}
  \KeywordTok{title}\NormalTok{(}\KeywordTok{paste}\NormalTok{(}\StringTok{"Degrees of freedom = "}\NormalTok{, }\KeywordTok{round}\NormalTok{(fit.ss}\OperatorTok{$}\NormalTok{df)),  }\DataTypeTok{outer =}\NormalTok{ F)}
  \KeywordTok{lines}\NormalTok{(water.grid, pred.ss}\OperatorTok{$}\NormalTok{y, }\DataTypeTok{lwd =} \DecValTok{2}\NormalTok{, }\DataTypeTok{col =} \StringTok{"blue"}\NormalTok{)}
\NormalTok{\}}
\end{Highlighting}
\end{Shaded}

\includegraphics{dsII_hw2_cs3652_files/figure-latex/unnamed-chunk-9-1.pdf}

\begin{Shaded}
\begin{Highlighting}[]
\CommentTok{#Use cross validation to find degrees of freedom }
\NormalTok{fit.ss <-}\StringTok{ }\KeywordTok{smooth.spline}\NormalTok{(concrete}\OperatorTok{$}\NormalTok{water, concrete}\OperatorTok{$}\NormalTok{compressive_strength)}
\NormalTok{fit.ss}\OperatorTok{$}\NormalTok{df}
\end{Highlighting}
\end{Shaded}

\begin{verbatim}
## [1] 68.88205
\end{verbatim}

\begin{Shaded}
\begin{Highlighting}[]
\NormalTok{pred.smooth <-}\StringTok{ }\KeywordTok{predict}\NormalTok{(fit.ss, }\DataTypeTok{x =}\NormalTok{ water.grid)}
\NormalTok{pred.smooth.df <-}\StringTok{ }\KeywordTok{data.frame}\NormalTok{(}\DataTypeTok{pred =}\NormalTok{ pred.ss}\OperatorTok{$}\NormalTok{y,}
                         \DataTypeTok{water =}\NormalTok{ water.grid)}

\NormalTok{fitplot <-}\StringTok{ }\KeywordTok{ggplot}\NormalTok{(}\DataTypeTok{data =}\NormalTok{ concrete, }\KeywordTok{aes}\NormalTok{(}\DataTypeTok{x =}\NormalTok{ water, }\DataTypeTok{y =}\NormalTok{ compressive_strength)) }\OperatorTok{+}
\StringTok{  }\KeywordTok{geom_point}\NormalTok{(}\DataTypeTok{color =} \KeywordTok{rgb}\NormalTok{(.}\DecValTok{2}\NormalTok{, .}\DecValTok{2}\NormalTok{, .}\DecValTok{4}\NormalTok{, .}\DecValTok{1}\NormalTok{)) }\OperatorTok{+}\StringTok{ }\KeywordTok{geom_line}\NormalTok{(}\KeywordTok{aes}\NormalTok{(}\DataTypeTok{x =}\NormalTok{ water, }\DataTypeTok{y =}\NormalTok{ pred), }\DataTypeTok{data =}\NormalTok{ pred.smooth.df, }
              \DataTypeTok{color =} \KeywordTok{rgb}\NormalTok{(.}\DecValTok{8}\NormalTok{, .}\DecValTok{1}\NormalTok{, .}\DecValTok{1}\NormalTok{, }\DecValTok{1}\NormalTok{)) }\OperatorTok{+}\StringTok{ }\KeywordTok{theme_bw}\NormalTok{() }
\NormalTok{fitplot}
\end{Highlighting}
\end{Shaded}

\includegraphics{dsII_hw2_cs3652_files/figure-latex/unnamed-chunk-10-1.pdf}

\begin{Shaded}
\begin{Highlighting}[]
\CommentTok{#fit 1}
\NormalTok{fit.ss1 <-}\StringTok{ }\KeywordTok{smooth.spline}\NormalTok{(concrete}\OperatorTok{$}\NormalTok{water, concrete}\OperatorTok{$}\NormalTok{compressive_strength, }\DataTypeTok{df =} \DecValTok{10}\NormalTok{)}
\NormalTok{fit.ss1}
\end{Highlighting}
\end{Shaded}

\begin{verbatim}
## Call:
## smooth.spline(x = concrete$water, y = concrete$compressive_strength, 
##     df = 10)
## 
## Smoothing Parameter  spar= 0.8687676  lambda= 0.001391258 (11 iterations)
## Equivalent Degrees of Freedom (Df): 10.00132
## Penalized Criterion (RSS): 81156.22
## GCV: 221.8629
\end{verbatim}

\begin{Shaded}
\begin{Highlighting}[]
\NormalTok{pred.ss1 <-}\StringTok{ }\KeywordTok{predict}\NormalTok{(fit.ss1, }\DataTypeTok{x =}\NormalTok{ water.grid)}
\NormalTok{pred.ss.df1 <-}\StringTok{ }\KeywordTok{data.frame}\NormalTok{(}\DataTypeTok{pred =}\NormalTok{ pred.ss1}\OperatorTok{$}\NormalTok{y, }\DataTypeTok{water =}\NormalTok{ water.grid)}
\CommentTok{#fit 2}
\NormalTok{fit.ss2 <-}\StringTok{ }\KeywordTok{smooth.spline}\NormalTok{(concrete}\OperatorTok{$}\NormalTok{water, concrete}\OperatorTok{$}\NormalTok{compressive_strength, }\DataTypeTok{df =} \DecValTok{30}\NormalTok{)}
\NormalTok{fit.ss2}
\end{Highlighting}
\end{Shaded}

\begin{verbatim}
## Call:
## smooth.spline(x = concrete$water, y = concrete$compressive_strength, 
##     df = 30)
## 
## Smoothing Parameter  spar= 0.5500421  lambda= 6.929549e-06 (11 iterations)
## Equivalent Degrees of Freedom (Df): 30.00244
## Penalized Criterion (RSS): 59933.93
## GCV: 208.9676
\end{verbatim}

\begin{Shaded}
\begin{Highlighting}[]
\NormalTok{pred.ss2 <-}\StringTok{ }\KeywordTok{predict}\NormalTok{(fit.ss2, }\DataTypeTok{x =}\NormalTok{ water.grid)}
\NormalTok{pred.ss.df2 <-}\StringTok{ }\KeywordTok{data.frame}\NormalTok{(}\DataTypeTok{pred =}\NormalTok{ pred.ss2}\OperatorTok{$}\NormalTok{y, }\DataTypeTok{water =}\NormalTok{ water.grid)}
\CommentTok{#fit 3}
\NormalTok{fit.ss3 <-}\StringTok{ }\KeywordTok{smooth.spline}\NormalTok{(concrete}\OperatorTok{$}\NormalTok{water, concrete}\OperatorTok{$}\NormalTok{compressive_strength, }\DataTypeTok{df =} \DecValTok{50}\NormalTok{)}
\NormalTok{fit.ss3}
\end{Highlighting}
\end{Shaded}

\begin{verbatim}
## Call:
## smooth.spline(x = concrete$water, y = concrete$compressive_strength, 
##     df = 50)
## 
## Smoothing Parameter  spar= 0.3836017  lambda= 4.347305e-07 (11 iterations)
## Equivalent Degrees of Freedom (Df): 50.00509
## Penalized Criterion (RSS): 46068.79
## GCV: 202.715
\end{verbatim}

\begin{Shaded}
\begin{Highlighting}[]
\NormalTok{pred.ss3 <-}\StringTok{ }\KeywordTok{predict}\NormalTok{(fit.ss3, }\DataTypeTok{x =}\NormalTok{ water.grid)}
\NormalTok{pred.ss.df3 <-}\StringTok{ }\KeywordTok{data.frame}\NormalTok{(}\DataTypeTok{pred =}\NormalTok{ pred.ss3}\OperatorTok{$}\NormalTok{y, }\DataTypeTok{water =}\NormalTok{ water.grid)}
\CommentTok{#cv fit}
\NormalTok{fit.ss <-}\StringTok{ }\KeywordTok{smooth.spline}\NormalTok{(concrete}\OperatorTok{$}\NormalTok{water, concrete}\OperatorTok{$}\NormalTok{compressive_strength)}
\NormalTok{fit.ss}
\end{Highlighting}
\end{Shaded}

\begin{verbatim}
## Call:
## smooth.spline(x = concrete$water, y = concrete$compressive_strength)
## 
## Smoothing Parameter  spar= 0.2578404  lambda= 5.371698e-08 (11 iterations)
## Equivalent Degrees of Freedom (Df): 68.88205
## Penalized Criterion (RSS): 36681.64
## GCV: 200.2892
\end{verbatim}

\begin{Shaded}
\begin{Highlighting}[]
\NormalTok{pred.ss <-}\StringTok{ }\KeywordTok{predict}\NormalTok{(fit.ss, }\DataTypeTok{x =}\NormalTok{ water.grid)}
\NormalTok{pred.ss.df <-}\StringTok{ }\KeywordTok{data.frame}\NormalTok{(}\DataTypeTok{pred =}\NormalTok{ pred.ss}\OperatorTok{$}\NormalTok{y, }\DataTypeTok{water =}\NormalTok{ water.grid)}
\CommentTok{#plot of fits}
\NormalTok{scatter <-}\StringTok{ }\KeywordTok{ggplot}\NormalTok{(}\DataTypeTok{data =}\NormalTok{ concrete, }\KeywordTok{aes}\NormalTok{(}\DataTypeTok{x =}\NormalTok{ water, }\DataTypeTok{y =}\NormalTok{ compressive_strength)) }\OperatorTok{+}
\StringTok{  }\KeywordTok{geom_point}\NormalTok{(}\DataTypeTok{color =} \KeywordTok{rgb}\NormalTok{(.}\DecValTok{2}\NormalTok{, .}\DecValTok{4}\NormalTok{, .}\DecValTok{2}\NormalTok{, .}\DecValTok{5}\NormalTok{))}
\NormalTok{smooth_fits <-}\StringTok{ }\NormalTok{scatter }\OperatorTok{+}\StringTok{ }
\StringTok{  }\KeywordTok{geom_line}\NormalTok{(}\KeywordTok{aes}\NormalTok{(}\DataTypeTok{x =}\NormalTok{ water, }\DataTypeTok{y =}\NormalTok{ pred), }\DataTypeTok{data =}\NormalTok{ pred.ss.df1,}
          \DataTypeTok{color =} \KeywordTok{rgb}\NormalTok{(.}\DecValTok{8}\NormalTok{, .}\DecValTok{1}\NormalTok{, .}\DecValTok{1}\NormalTok{, }\DecValTok{1}\NormalTok{)) }\OperatorTok{+}\StringTok{ }
\StringTok{  }\KeywordTok{geom_line}\NormalTok{(}\KeywordTok{aes}\NormalTok{(}\DataTypeTok{x =}\NormalTok{ water, }\DataTypeTok{y =}\NormalTok{ pred), }\DataTypeTok{data =}\NormalTok{ pred.ss.df2,}
          \DataTypeTok{color =} \KeywordTok{rgb}\NormalTok{(}\DecValTok{0}\NormalTok{, }\DecValTok{0}\NormalTok{, }\DecValTok{1}\NormalTok{, }\DecValTok{1}\NormalTok{)) }\OperatorTok{+}\StringTok{ }
\StringTok{  }\KeywordTok{geom_line}\NormalTok{(}\KeywordTok{aes}\NormalTok{(}\DataTypeTok{x =}\NormalTok{ water, }\DataTypeTok{y =}\NormalTok{ pred), }\DataTypeTok{data =}\NormalTok{ pred.ss.df3,}
          \DataTypeTok{color =} \KeywordTok{rgb}\NormalTok{(}\DecValTok{1}\NormalTok{, }\DecValTok{0}\NormalTok{, }\DecValTok{1}\NormalTok{, }\DecValTok{1}\NormalTok{)) }\OperatorTok{+}\StringTok{ }
\StringTok{  }\KeywordTok{geom_line}\NormalTok{(}\KeywordTok{aes}\NormalTok{(}\DataTypeTok{x =}\NormalTok{ water, }\DataTypeTok{y =}\NormalTok{ pred), }\DataTypeTok{data =}\NormalTok{ pred.ss.df,}
          \DataTypeTok{color =} \KeywordTok{rgb}\NormalTok{(.}\DecValTok{2}\NormalTok{, .}\DecValTok{2}\NormalTok{, .}\DecValTok{4}\NormalTok{, }\DecValTok{1}\NormalTok{)) }\OperatorTok{+}\StringTok{ }
\StringTok{  }\KeywordTok{theme_bw}\NormalTok{()}
\NormalTok{smooth_fits}
\end{Highlighting}
\end{Shaded}

\includegraphics{dsII_hw2_cs3652_files/figure-latex/unnamed-chunk-11-1.pdf}

\section{Part D}\label{part-d}

\begin{Shaded}
\begin{Highlighting}[]
\NormalTok{gam.m1 <-}\StringTok{ }\KeywordTok{gam}\NormalTok{(compressive_strength }\OperatorTok{~}\StringTok{ }\NormalTok{cement }\OperatorTok{+}\StringTok{ }\NormalTok{blast_furnace_slag }\OperatorTok{+}\StringTok{ }\NormalTok{fly_ash }\OperatorTok{+}\StringTok{ }\NormalTok{water }\OperatorTok{+}\StringTok{ }\NormalTok{superplasticizer }\OperatorTok{+}\StringTok{ }\NormalTok{coarse_aggregate }\OperatorTok{+}\StringTok{ }\NormalTok{fine_aggregate }\OperatorTok{+}\StringTok{ }\NormalTok{age, }\DataTypeTok{data =}\NormalTok{ concrete)}

\NormalTok{gam.m2 <-}\StringTok{ }\KeywordTok{gam}\NormalTok{(compressive_strength }\OperatorTok{~}\StringTok{ }\NormalTok{cement }\OperatorTok{+}\StringTok{ }\NormalTok{blast_furnace_slag }\OperatorTok{+}\StringTok{ }\NormalTok{fly_ash }\OperatorTok{+}\StringTok{ }\KeywordTok{s}\NormalTok{(water) }\OperatorTok{+}\StringTok{ }\NormalTok{superplasticizer }\OperatorTok{+}\StringTok{ }\NormalTok{coarse_aggregate }\OperatorTok{+}\StringTok{ }\NormalTok{fine_aggregate }\OperatorTok{+}\StringTok{ }\NormalTok{age, }\DataTypeTok{data =}\NormalTok{ concrete)}

\KeywordTok{anova}\NormalTok{(gam.m1, gam.m2, }\DataTypeTok{test =} \StringTok{"F"}\NormalTok{)}
\end{Highlighting}
\end{Shaded}

\begin{verbatim}
## Analysis of Deviance Table
## 
## Model 1: compressive_strength ~ cement + blast_furnace_slag + fly_ash + 
##     water + superplasticizer + coarse_aggregate + fine_aggregate + 
##     age
## Model 2: compressive_strength ~ cement + blast_furnace_slag + fly_ash + 
##     s(water) + superplasticizer + coarse_aggregate + fine_aggregate + 
##     age
##   Resid. Df Resid. Dev     Df Deviance      F   Pr(>F)    
## 1    1021.0     110413                                    
## 2    1013.4     106140 7.5562   4272.8 5.4038 2.01e-06 ***
## ---
## Signif. codes:  0 '***' 0.001 '**' 0.01 '*' 0.05 '.' 0.1 ' ' 1
\end{verbatim}

\begin{Shaded}
\begin{Highlighting}[]
\KeywordTok{plot}\NormalTok{(gam.m2)}
\end{Highlighting}
\end{Shaded}

\includegraphics{dsII_hw2_cs3652_files/figure-latex/unnamed-chunk-13-1.pdf}

\begin{Shaded}
\begin{Highlighting}[]
\KeywordTok{vis.gam}\NormalTok{(gam.m2, }\DataTypeTok{view =} \KeywordTok{c}\NormalTok{(}\StringTok{"water"}\NormalTok{, }\StringTok{"fine_aggregate"}\NormalTok{), }
        \DataTypeTok{plot.type =} \StringTok{"contour"}\NormalTok{, }\DataTypeTok{color =} \StringTok{"topo"}\NormalTok{)}
\end{Highlighting}
\end{Shaded}

\includegraphics{dsII_hw2_cs3652_files/figure-latex/unnamed-chunk-13-2.pdf}

\begin{Shaded}
\begin{Highlighting}[]
\KeywordTok{vis.gam}\NormalTok{(gam.m2, }\DataTypeTok{view =} \KeywordTok{c}\NormalTok{(}\StringTok{"water"}\NormalTok{,}\StringTok{"coarse_aggregate"}\NormalTok{),}
        \DataTypeTok{plot.type =} \StringTok{"contour"}\NormalTok{, }\DataTypeTok{color =} \StringTok{"topo"}\NormalTok{)}
\end{Highlighting}
\end{Shaded}

\includegraphics{dsII_hw2_cs3652_files/figure-latex/unnamed-chunk-13-3.pdf}


\end{document}
